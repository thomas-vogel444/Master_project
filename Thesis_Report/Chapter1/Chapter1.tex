\chapter{Introduction}

\section{Deep Learning and Convolution Neural Networks}

\noindent Deep Learning is a branch of Machine Learning referring to neural networks with many hidden layers. It has attracted a lot of attention lately \citep{NYT_article} due to a string of recent successes, which has made it the state of the art in a number of international contests in pattern recognition, particularly in image recognition, speech recognition, and natural language processing \citep{Deep_Learning_overview}. \\

\noindent The main difficulty in applying Machine Learning techniques to real world problems comes from the high dimensionality of the datasets encountered, severely increasing the learning complexity of the algorithms required to handle them and thereby reducing their ability to generalise well. Traditionally, practitioners tackle this problem by finding ways to reduce the number of dimensions in what is known as "feature extraction", a deliberate hand-engineered process of transforming the large number of variables into a smaller number of features with the same level of information. By contrast, neural networks aim to learn the relevant features as part of the learning process itself using a hierarchical network of simple processing units called neurons, or nodes, in order to extract a good representation of the data \citep{Bengio:2009:LDA:1658423.1658424}. Learning in this context means finding a set of parameters known as weights which makes the neural network exhibit a particular desired behaviour.\\

\noindent Neural networks are not new, but until recently, deep multi-layered architectures couldn't be trained effectively using standard gradient descent methods based on the backpropagation algorithm from random initialisation (WHY?). In 2006, a seminal paper by Hinton \citep{Hinton:2006:FLA:1161603.1161605} made the training of deep neural networks practical using an unsupervised pre-training approach. These networks called Deep Belief Networks, have lead to a resurgence in the interest in neural networks. Since then, a number of other practical mechanisms have been devised to train deep architectures without unsupervised pre-training \citep{Larochelle:2009:EST:1577069.1577070}, from more effective initialisation procedures \citep{Glorot10understandingthe} to the use of Rectified Linear activation function \citep{AISTATS2011_GlorotBB11} or the introduction of dropout as a regularisation mechanism \citep{JMLR:v15:srivastava14a}.\\

\noindent Another major recent development was the introduction of convolutional neural networks (CNNs) by LeCun \citep{Lecun98gradient-basedlearning}. CNNs are a specialised type of neural network for certain grid-like structures with localised features. In particular, they have been remarkably effective for classification tasks on images, videos, and audio recordings, providing state of the art performance on object and speech recognition tasks \citep{NIPS2010_4133}, \citep{going_deeper_with_convolutions}, \citep{NIPS2012_4824}.\\

\section{Deep Learning in Medical Imaging}

\noindent The field of Medical Imaging encompasses a set of techniques to create visual representations of the interior of the body using technologies such as X-rays or ultra-sounds. Its aim is to provide non-invasive ways of revealing internal structures to help make diagnoses. Interpreting these images have traditionally been done by trained clinicians such as radiologist or histologists, involving a time consuming and expensive process. Machine Learning is now being used to help clinicians with this process \citep{key:article} in order to improve the accuracy, speed and cost of diagnoses.\\

\noindent Following the string of recent success of Deep Learning approaches in various imaging contexts, there has been recent efforts to implement these classifiers in the context of Medical Imaging. They include for instance the anatomical segmentation of the entire brain \citep{DBLP:journals/corr/BrebissonM15}, biological neuron membrane to map 3D brain structure and connectivity \citep{NIPS2012_4741}, and counting cell mitosis in histology images \citep{Ciresan} for cancer screening and assessment, amongst others.\\

\section{Goal of the Thesis}

This Master's thesis aims to implement a CNN to identify the atrium of the heart by classifying each voxel as either being part of it or outside of it using a tri-planar multi-scale approach. The second chapter discusses some background material on neural networks and CNNs. The third chapter then describes the approach undertaken and presents a number of experimental results. We will finish with some conclusions and a discussion of various ways to improve upon our results.










