\documentclass[11pt,a4,twosided,singlespacing,titlepagenumber=on]{scrreprt}
%%%obsolete starting points. (ignore)
%\documentclass[a4paper,12pt,singlespace,report]{memoir}
%\documentclass[a4,11pt,singlespace,MSc]{icldt}

\usepackage[T1]{fontenc} % Handles accents etc better in the invisible details of the pdf output.
\usepackage[latin1]{inputenc} % May or may not be needed. Says that your *.tex file is a text file with ASCII latin1 encoding. You could use e.g. utf8 instead for easier accents etc.
\usepackage[UKenglish]{babel} % Let LaTeX know what language the text is in so it can select the correct hyphenation pattern etc

%%% American Mathematical Society packages
\usepackage{amsfonts,amssymb,amsmath,amsthm}
\usepackage{amsbsy}

%%% Graphics packages
\usepackage{float}
%\floatstyle{boxed} 
\restylefloat{figure}
\usepackage{graphicx}
%\graphicspath{{figures/}} % Useful if you have lots of images and want to keep thinks tidy by having a subfolder for images
%\usepackage{tikz} %For creating vector-graphics diagrams, flowcharts etc directly in LaTeX (takes some time to learn)
\usepackage[absolute]{textpos} % Used to position the Imperial College logo. You can comment this line and the next line out if you don't use the logo.
\usepackage[table,xcdraw]{xcolor}

%%% Referencing and cross-referencing
\usepackage[colorlinks=false,pdfborder={0 0 0},plainpages=false,pdfpagelabels]{hyperref} % If you click on an item in the table of contents or a referenced equation/figure number, the PDF will go to the desired page. Neat isn't it?

%\usepackage[round,authoryear,sort]{natbib} % Enable bibtex-based bibliography generation
%\usepackage{cite}
\usepackage[square,numbers,sort&compress]{natbib} % If you want numbered referencing instead of author-year style.

%\setcounter{secnumdepth}{3} %If you want subsubsections to be numbered
\numberwithin{equation}{chapter} % Reset equation numbers after each chapter.

\usepackage{hyperref}

\usepackage{listings}
\usepackage{color}

\definecolor{dkgreen}{rgb}{0,0.6,0}
\definecolor{gray}{rgb}{0.5,0.5,0.5}
\definecolor{mauve}{rgb}{0.58,0,0.82}

\lstdefinestyle{myLuastyle}
{frame=tb,
  language={[5.0]Lua},
  aboveskip=3mm,
  belowskip=3mm,
  showstringspaces=false,
  columns=flexible,
  basicstyle={\small\ttfamily},
  numbers=none,
  numberstyle=\tiny\color{gray},
  keywordstyle=\color{blue},
  commentstyle=\color{dkgreen},
  stringstyle=\color{mauve},
  breaklines=true,
  breakatwhitespace=true,
  tabsize=3
}

\usepackage[toc,page]{appendix}


%%%%%%%%%%%%%%%%%%%%%%%%%%%%%%%%%%%%%
%%%%% Define how to create the title page  %%%%%%%%%%%%%%%%
%%%%%%%%%%%%%%%%%%%%%%%%%%%%%%%%%%%%%
\makeatletter
\newcommand*{\supervisor}[1]{\gdef\@supervisor{#1}}
\newcommand*{\CID}[1]{\gdef\@CID{#1}}
\newcommand*{\logoimg}[1]{\gdef\@logoimg{#1}}
\renewcommand{\maketitle}{
\begin{titlepage}
\ifdefined\@logoimg
\begin{textblock*}{8cm}(1.75cm,1.75cm)
\includegraphics[width=70mm]{\@logoimg}
\end{textblock*}
\vspace*{1cm}
\else
%\vspace*{0cm}
\fi
\begin{center}
\vspace*{\stretch{0.1}}
Imperial College London\\
Department of Mathematics\par
\vspace*{\stretch{1}} % This inserts vertical space and allows you to specify a relative size for the vertical spaces.
{\titlefont\Huge \@title\par} % If your title is long, you may wish to use \huge instead of \Huge.
\vspace*{\stretch{2}}
{\Large \@author \par}
\vspace*{1em}
{\large CID: \@CID \par}
\vspace*{\stretch{0.5}}
{\large Supervised by \@supervisor \par}
\vspace*{\stretch{3}}
{\Large \@date \par}
\vspace*{\stretch{1}}
{\large Submitted in partial fulfilment of the requirements for the
MSc in Statistics of Imperial College London}
\vspace*{\stretch{0.1}}
\end{center}%
\end{titlepage}%
}
\makeatother

%%% And the plagiarism declaration
\newcommand*{\declaration}{%
\vspace*{0.3\textheight}
The work contained in this thesis is my own work unless
otherwise stated.\\
\vspace*{0.1\textheight}\\
\hspace*{0.25\textwidth}Signed: \hspace{0.25\textwidth} Date:
\clearpage}

%%% And the abstract page
\renewenvironment{abstract}%
{\chapter*{Abstract}\thispagestyle{plain}}%
{\clearpage}
%%% And why not change the quote environment
\newenvironment{myquote}%
{\begin{quote}{\Large{}``}}%
{\ifhmode\unskip\fi{\Large{}''}\end{quote}}




%%% Actual words used in the title page
\title{Segmentation of CT scans into Atrium/non Atrium}
\author{Thomas Vogel}
\CID{01016217}
\supervisor{Prof Giovanni Montana}
\date{\today}
\logoimg{Imperial__4_colour_process.jpg}

%%%%%%%%%%%%%%%%%%%%%%%%%%%%%%%%%%%%%
%%%%% End of preamble and start of document %%%%%%%%%%%%%%
%%%%%%%%%%%%%%%%%%%%%%%%%%%%%%%%%%%%%

\begin{document}

\maketitle %Generates the Title Page

\declaration %Insert plagiarism statement

\begin{abstract}
The object of this thesis is to implement a convolutional neural network (CNN) to automatically identify the atrium of the heart in Computerised Tomography (CT) scans of the chest. A brief overview of Deep Learning is given in general, and in the context of Medical Imaging in particular as well as some background material on CNNs. We then describe the implementation of our CNN using a multi-scale tri-planar approach followed by some experimental results from a hyper-parameter search for a good architecture and set of learning parameters. The final set of results describes an altering to the sampling procedure yielding a significant improvement to the classification rate. The final model has an mean classification accuracy of 0.985 across 7 test CT scans, but makes significant errors in the neighbouring regions around the atrium where it is expected and, more worryingly, at times also away from it.
\end{abstract}
\newpage
\chapter*{Acknowledgements}
I would like to thank Professor Giovanni Montana for supplying the computational resources needed for this project as well as Rudra Poudel, his research associate, for providing technical help when I needed it. This thesis was written during a summer riddled with personal difficulties including a severe sciatica which severely threatened its completion. Special acknowledgement goes to my mother, who nursed me for 6 weeks in July and August while I was bed bound and unable to walk, and to fellow student Irina Timoshenko for helping me overcome a number of deep psychological issues.
\newpage

% Automatically create a table of contents
\renewcommand{\contentsname}{Table of Contents}
\tableofcontents
\newpage

% Figure and table lists if you want them.
%\cleardoublepage
%\phantomsection
%\listoffigures 
%\addcontentsline{toc}{chapter}{\listfigurename}
%\newpage
%\phantomsection
%\listoftables  
%\addcontentsline{toc}{chapter}{\listtablename}
%\newpagex

\chapter{Introduction}

\section{Deep Learning and Convolution Neural Networks}

\noindent Neural networks, or Deep Learning as it is now referred, is a field of artificial intelligence that has attracted a lot of attention recently \citep{NYT_article}. Over the last few years, they have become state of the art in a number of international contests in pattern recognition, particularly image recognition, speech recognition, and natural language processing\citep{Deep_Learning_overview}. \\

\noindent Most of the difficulties in applying Machine Learning techniques to real world problems comes from the high dimensionality of the datasets encountered, severely increasing the learning complexity of the algorithms required to handle them and thereby reducing their generalisation properties. Traditionally this problem is tackled by finding ways to reduce the number of dimensions in what is known as "feature extraction", a deliberate hand-engineered process of transforming the large number of variables into a smaller number of features with the same level of information content. By contrast, neural networks aim to learn the relevant features as part of the learning process itself, by training a hierarchical network of simple processing units called neurons or nodes on a large set of data in order to extract a good representation model of the data. Learning in this context means finding a set of weights which makes the neural network exhibit the desired behaviour (e.g. classify hand written-digits correctly).\\

\noindent The first neural networks with multiple, if few, layers date back to the 60s [reference] and have been around for decades. For feedforward neural networks, where the information can only one way through the network, an efficient gradient based method that is widely used called the backpropagation method was developed then and applied to neural networks in 1981 [reference]. However a number of practical difficulties made it difficult to train architectures with more a small number of layers. In particular, one common issue was the so-called neuron saturation problem [reference] where I need to explain what that means.A number of practical solutions where offered [reference], from the choice of parameter initialisation to the influence of the various possible activation functions on the learning behaviour of the network [reference]. However in 2006, a seminal paper [reference] by Hinton made training deep neural networks practical using an unsupervised pre-training approach. These networks called Deep Belief Networks, have lead to a resurgence in the interest in neural networks.\\

\noindent Another major recent development was the introduction of convolutional neural networks (CNNs) by LeCun [reference]. CNNs are a specialised type of Neural Network for certain grid-like structures with features that are strongly localised. In particular, they have been remarkably effective on classification tasks on images, videos, and audio recordings, providing state of the art performance on object and speech recognition tasks. They were introduced in 1998 in the context of hand-written digit recognition on the MNIST dataset. Ever since they have won many official international pattern recognition competitions, in particular the Imagenet challenge. A full historical account can be found in [reference].\\

\section{Deep Learning in Medical Imaging}

Medical Imaging is a set of techniques to create visual representations of the interior of the body using technologies such as X-rays or ultra-sounds. It's aim is to provide a non-invasive way of making diagnosis by revealing internal structures. Interpreting these images have traditionally been done by trained clinicians such as radiologist or histologists, involving a time consuming and expensive process. Current efforts are being made to automate this process [reference] in order to improve the accuracy, speed and cost of diagnoses.\\

Following the string of recent success of Deep Learning approaches in the various contexts, there has been a recent effort to implement these models in the context of Medical Imaging. These efforts include for instance the anatomical segmentation of the entire brain [reference], the segmentation of the lungs for cancer detection [reference], biological neuron membrane to map 3D brain structure and connectivity [reference], tibial cartilage [reference], detection of bone tissue in X-ray images [reference] and counting cell mitosis in histology images [reference] for cancer screening and assessment, amongst others.\\

The segmentation process represents a first step necessary for any automatic method of extracting information from an image. The machine learning approach is to train a model to classify each voxel into its corresponding anatomical region given a training set consisting of labelled medical images.\\

Automating this task would enable systematic segmentation of medical images on the fly as soon as these images are acquired. They could also be useful in the detection of anatomical abnormalities such in tumour detections.\\

\section{Goal of the Thesis}

This Master's thesis aims to segments a 

takes direct inspiration from the paper on brain segmentation, where the entire brain was segmented into 140 different anatomical regions. It will propose a deep convolution neural network for the automated segmentation of chest CT scans into atrium and non-atrium parts using a tri-planar multi-scale approach. \\

The second chapters discusses some background on neural networks and CNNs. The second chapter then describes the approach undertaken and presents a number of experimental results from model selection, the varying of dataset size and various sampling strategies. We will finish with some conclusions and a discussion of various other ways to improve upon our results.











\chapter{Background Material: on Convolutional Neural Networks}

\noindent We first start with a brief review of feedforward neural networks before elaborating on convolutional neural networks (CNNs). A number of good books available expand on the following content in much greater detail, notably \citep{Bengio-et-al-2015-Book}, \citep{Bishop:1995:NNP:525960} and \citep{Nielsen}.

\section{Feed Forward Neural Networks}

\begin{figure}
\centering
\label{Basic_neural_network}
%\includegraphics[trim=2cm 2cm 2cm 2cm, clip=true, height=80mm]{Chapter2/FF_Neural_Network.pdf}
\includegraphics[trim=4cm 17cm 3cm 4cm, clip=true, height=80mm]{Tikz/NN.pdf}
\caption{A 1 layer-feedforward neural network with 4 input nodes, 1 hidden layer with hidden nodes and 2 output node.}
\end{figure}

A feedforward neural network is a standard model in the machine literature for classification problems. It consists of a number of layers composed of simple processing units, also called neurons or nodes, in which each unit in a given layer is connected to a number of units in the previous layers, each connection characterised by a weight encoding the knowledge of the network. Layers in which every unit is only connected to every unit in the previous layer are called fully-connected layers. Each unit represents an activation value generated by passing a linear combination of the values of the incoming connections weighted by their weights to an activation function, such as a Rectified Linear, a Tanh or a Sigmoid function. The information encoded in the data enters at the input layer, gets processed as it passes through the network, layer by layer, until it arrives at the output layer. The choice of the activation function at the output layer is determined by the nature of the data and the assumed distribution of the target variables. For classification problems, the softmax function or, in our case, its log, give the output a probabilistic interpretation. This model can be represented in the form of a network diagram as shown in Figure \ref{Basic_neural_network}.\\

\noindent Training a neural network is done by minimising an error measure of its performance over a training set using gradient-based optimisation routines. The error gradients with respect to the parameters that are needed for the minimisation procedure, can be efficiently evaluated via the standard backpropagation algorithm. To prevent overfitting, a number of regularisation methods are available, including the traditional ones such as L1 and L2 regularisation. Recently a method called dropout \citep{JMLR:v15:srivastava14a} where at every epoch, a percentage of hidden units are randomly deactivated during training has been shown to give better generalisation performance.

\section{Convolutional Neural Networks}

\noindent A CNN is a specialised kind of feedforward neural network where the first few layers of the architecture are so-called convolutional layers. These consist of three stages: a convolution stage, a detection stage and an optional subsampling stage. We will be discussing CNNs in the context of multi-channel images of size $m \times m$, where each pixel value represents an input node. 

\subsection{Convolution}

\begin{figure}
\centering
\label{convolution_operation}
\includegraphics[trim=2cm 7cm 2cm 7cm, clip=true, height=80mm]{Chapter2/convolution.pdf}
\caption{The convolution layer. Illustration of the formation of a feature map.}
\end{figure}

\noindent The convolution stage is responsible for implementing $k$ convolution operations over each channel of the input image. This is accomplished by convolving in parallel each channel with $k$ kernels of size $s \times s$. Unlike fully connected layers, in a convolution, each output node is locally connected to a small square subset of corresponding input nodes determined by the kernel size. Additionally, as the same kernel is applied throughout the image, the convolution operation implies weight sharing, thus greatly reducing the number of trainable parameters. Figure \ref{convolution_operation} illustrates the convolution operation graphically. The set of output nodes resulting from one kernel is called a feature map, which is itself a rectangular arrays of nodes of size $(m - s/2 + 1) \times (m - s/2 + 1)$. Feature maps detect the presence in the input image of a particular feature encoded by the corresponding kernel. Having multiple kernels run through the input layer in parallel produces a set of feature maps. Together, they are responsible for detecting various types of features which might be present in the image. \\

\noindent The benefits of this approach are two folds :

\begin{itemize}
	\item Computational efficiency: the sparse interaction between the hidden nodes and the weight sharing significantly decrease the number of parameters to train.
	\item Translational equivariance: shifting the input results in an equivalent shift in the output. This is a property of the convolution operation.
\end{itemize}

\noindent In the detector stage, every node in the feature map is then passed to an nonlinear activation unit exactly as in a standard neural network. These activation units usually consist of either Rectified Linear units (ReLU), Tanh units, or Sigmoid units.\\

\begin{figure}
\centering
\label{subsampling}
\includegraphics[trim=0cm 0cm 0cm 0cm, clip=true, height=60mm]{Chapter2/pooling.pdf}
\caption{Subsampling in action. Here maxpooling is used to halve the size of an feature map using a $2 \times 2$ filter}
\end{figure}

\noindent Finally in the pooling stage, each feature map is then subsampled by aggregating the node values using a summary statistic over a rectangular neighbourhood of outputs. Two commonly used ones are the max and average pooling operations which report respectively the maximum and average of a set of inputs. Figure \ref{subsampling} shows a diagram of subsampling.\\

\noindent The benefits of subsampling are that it further reduces the number of features by a factor of k for pooling regions spaced k pixels apart, aiding the classification task and improving the computational efficiency of the network. Furthermore subsampling has the added benefit of making the representation become invariant to small translations of the input. Thus translating the input by a small amount results in little to no change in the values of the pooled outputs.\\

\subsection{Typical Architecture}

\noindent A typical convolutional neural network architecture consists of an input layer, a number of convolutional layers, a number of fully connected layers, followed by an output layer. The input layer is an $r \times n \times n$ image, with $r$ being the number of channels and $n$ the image height and width. Each channel is passed through a number convolutional layer in parallel. Then the remaining output nodes from all the resulting feature maps across all channels are "flattened" and passed as input to a number of fully connected layers followed by an output layer. The convolutional layers thus serve as a way to reduce the dimensionality of the image by extracting meaningful local features.








\chapter{Experimental Results}

\section{The Data}

\noindent The aim of this thesis is to implement, and then fine-tune, a CNN to classify voxels of chest computerised tomography (CT) scan as being either in the part of the heart called the atrium or outside of it. The atrium is the two entry point of the blood into the heart. It is composed of two chambers: a right one which recovers blood returning to the heart to complete the cardiovascular cycle through the body, and a left one receiving blood coming back from the lungs after being oxygenated and purified of toxins.\\

\noindent The data from which the training and testing datasets are generated comprise 27 CT scans. CT scans are 3 dimensional grey scale images, generally of size 480*480*50, generated via computers combining many X-ray images taken from different angles to produce cross-sectional images of specific body parts. They are stored as DICOM files, DICOM standing for Digital Imaging and Communications in Medicine which is a standard for handling, storing, printing, and transmitting information in medical imaging. Each has been segmented by trained radiologists at St Thomas's hospital, the results of which are stored in Nearly Raw Raster Data (NRRD) files, a standard format for storing raster data. These are 3D arrays of the same dimension as their corresponding CT scan with each entry being either a 1 or 0 indicating whether its corresponding voxel belongs in the atrium or not. 

\section{Generating the datasets: The Tri-Planar Method}

\begin{figure}
\centering
\label{tri-planar}
\begin{minipage}{0.45\textwidth}
\centering
\includegraphics[trim=3cm 1.5cm 3cm 1.5cm, clip=true, height=60mm]{Chapter3/example_slice.png}
\end{minipage}\hfill
\begin{minipage}{0.45\textwidth}
\centering
\includegraphics[trim=2cm 8cm 2cm 8cm, clip=true, height=60mm]{Chapter3/triplanar.pdf}
\end{minipage}
\caption{Left: grey scale slices from a CT scan taken in the transversal, saggital and coronal planes. Right: illustration of the triplanar}
\end{figure}

\noindent Classifying the voxels requires building a set of input vectors each containing enough local and global information to allow the neural network to learn effectively. One way of providing 3 dimensional information is to use the tri-planar method. This consists in generating 3 perpendicular square patches of a given dimension in the transversal, saggital and coronal planes centred at the voxel of interest as shown in Figure \ref{tri-planar}. This technique has been found to give competitive results compared to using 3D patches while being much more computationally and memory efficient \citep{knee_cartilage}. In addition, we use a multi-scale approach as in \citep{DBLP:journals/corr/BrebissonM15}, where we add 3 more input channels composed of compressed patches that are originally 5 times larger than the above set of patches, but resized to be of the same size as the first 3 input patches to provide global information about the surroundings of the concerned voxel.\\

\noindent Each patch is fed into a different input channel of the CNN to a total of 6 channels. The outputs of those channels are then feed as inputs to a set of connected layers itself connected to a classifying layer.

\section{Implementation Details}

\subsection{Libraries}

\noindent Our CNNs were implemented using Torch, an open-source library aiming to provide a Matlab-like environment for scientific computing in Lua, along with a number of dependent libraries (nn, cunn, cudnn, fbcunn) which facilitate the training of neural networks on single and multiple GPUs. In addition, we use Python and a number of its libraries to handle all the logistics ranging from generating the datasets to producing plots of segmentation results.

\subsection{Computer facilities}

\noindent The training of the CNNs were conducted alternatively on one of 2 multi-GPU clusters, named montana-nvidia and montana-k80, kindly provided by Prof. Montana. montana-nvidia consists of 24 cores with 129 Gb of memory connected to two NVIDIA Tesla K40m and two Tesla K20Xm while montana-k80 on the other hand has 56 cores with 258 Gb of memory supported by 8 of NVIDIA's Testla K80. 

\section{Model Selection}

\subsection{General approach for model selection}

\noindent The purpose of this section is to find a good set of hyper-parameters fro our CNN. To that end, we allocate 20 of the 27 CT scans for generating the training set and the remaining 7 for generating the validation set. The training set is composed of 400000 training examples equally divided among the training CT scans, half being in the atrium and the other half outside it. The validation set is composed of 200000 examples equally divided among the testing CT scans. Furthermore at the end of training, each model fully segments a test CT scan from the testing set to provide performance statistics. In particular, from this segmentation image, we evaluate the model's sensitivity (the percentage of correctly classified atrium voxels), specificity (the percentage of correctly classified non-atrium voxels), and overall classification rate also known as the Dice coefficient, calculated by evaluating the proportion of correctly classified voxels in the segmented image. As 98\% of the CT scan is composed of non-atrium voxels, the Dice coefficient is overwhelmingly influenced by the specificity. We will conduct our model selection by selectively choosing hyper-parameters that optimise the Dice coefficient while having a reasonable sensitivity in the regions of 80 to 90\%.\\

Additionally, mask images are generated from 3 transversal slices in the segmented CT. The masks are formed by overlaying the grey scale CT scan image with colours representing the error status of the classification of a given voxel. They provide a visual representation of the performance of the models. The colour codes are:

\begin{itemize}
	\item Green: correctly classified atrium voxel.
	\item Blue: correctly classified non-atrium voxel.
	\item Red: incorrectly classified atrium voxel.
	\item Pink: incorrectly classified non-atrium voxel.
\end{itemize}

\noindent We start off our model selection with a CNN comprised of an input layer with 6 channels of patches of size 32*32, 2 convolutional layers with 32 and 64 feature maps respectively and a max pooling filter of size $2 \times 2$, 2 fully connected layers with 1000 and 500 hidden units each respectively, and a logsoftmax output layer giving the log probabilities of the voxel belonging to either classes. The training parameters are composed of a learning rate of 0.01, a momentum rate set to 0, and a mini-batch size set to 6000 examples. In addition, the negative log likelihood criterion provides the error measure during training. We will be varying in turn the following hyper-parameters while at each stage keeping the others constant.\\

\begin{itemize}
	\item the number of convolutional layers.
	\item the number of connected layers.
	\item the number of feature maps in the chosen number of convolutional layers.
	\item the number of hidden units in the chosen number of connected layers.
	\item the learning rate.
	\item the momentum.
	\item the type of activation function (ReLU, Tanh or Sigmoid).
\end{itemize}

\subsection{Varying the Number of Convolutional Layers}

\noindent To select the number of convolutional layers, we train 4 CNNs with architectures starting with the following convolutional layers:

\begin{itemize}
	\item Input (6*32*32) => Conv layer (32*28*28) => 2*2 MaxPooling filter (32*14*14) 
	\item Input (6*32*32) => Conv layer (32*28*28) => 2*2 MaxPooling filter (32*14*14) => Conv layer (64*10*10) => 2*2 MaxPooling filter (64*5*5)
	\item Input (6*32*32) => Conv layer (32*28*28) => Conv layer (32*24*24) => 2*2 MaxPooling filter (32*12*12) => Conv layer (64*8*8) => 2*2 MaxPooling filter (64*4*4)
	\item Input (6*32*32) => Conv layer (32*28*28) => Conv layer (32*24*24) => 2*2 MaxPooling filter (32*12*12) => Conv layer (64*8*8) => Conv layer (64*4*4) => 2*2 MaxPooling filter (64*2*2)
\end{itemize}

\noindent The values in parentheses indicate the number and dimensions of the feature maps at each layer. The following table gives the results for each of these architectures trained over 100 epochs.\\

\begin{tabular}{cccccc}
\rowcolor[HTML]{C0C0C0} 
        \# Conv Layers & \begin{tabular}[c]{@{}c@{}}Dice \\ training set\end{tabular} & \begin{tabular}[c]{@{}c@{}}Dice \\ testing set\end{tabular} & Sensitivity & Specificity & \begin{tabular}[c]{@{}c@{}}Dice \\ test CT scan\end{tabular} \\ \hline
1  & 0.959                                                        & 0.956                                                       & 0.913       & 0.970       & 0.969                                                        \\ 
2  & 0.958                                                        & 0.961                                                       & 0.734       & 0.989       & 0.984                                                        \\ 
3  & 0.956                                                        & 0.960                                                       & 0.762       & 0.987       & 0.983                                                        \\ 
4  & 0.959                                                        & 0.956                                                       & 0.881       & 0.973       & 0.972                                                       
\end{tabular}\\

\noindent The table above presents the key test statistics across the 4 architectures considered. The architecture with the best test Dice coefficient is the one with 2 convolutional layer, closely followed by the one with 3 convolutional layers, with dice coefficients of 0.984 and 0.983 respectively. These two architectures give low sensitivities of 0.734 and 0.762 respectively. However the architectures with 1 and 4 convolutional layers yield significantly lower Dice coefficients of 0.969 and 0.972 respectively, despite a much greater sensitivity of 0.913 and 0.881. \\

\begin{figure}
\centering
\includegraphics[trim=2.5cm 1.5cm 2cm 1.5cm, clip=true, height=80mm, width=150mm]{Chapter3/mask_results_varying_number_of_convolutional_layers.png}
\caption{Masks for varying number of convolutional layers}
\end{figure}

\noindent Figure whatever brings a visual element to this comparison. The first and fourth columns correspond to segmentations by the architectures with 1 and 4 convolutional layers respectively. These show much larger pink patches, corresponding to higher rates of false positives than the other two columns but with smaller red regions corresponding to lower rates of false negatives, corroborating the story told by the sensitivity and specificity in the table above. There is not much difference between the masks of the second and third layers.\\

\noindent We choose to go with the architecture with the highest Dice coefficient at this stage and thus select an architecture with 2 convolutional layers. 

\subsection{varying number of connected layers}

\noindent Having settled on an architecture with 2 convolutional layers, we now train 3 CNNs with 1, 2, and 3 fully connected layers each starting with 1000 hidden units and halving the number of hidden units for each additional layer. Hence the CNN with 3 fully connected layers has 1000, 500, and 250 hidden units in each layer respectively. The results are shown in the following table.\\

\begin{tabular}{cccccc}
\rowcolor[HTML]{C0C0C0} 
             \begin{tabular}[c]{@{}c@{}}\# Connected \\ Layers \end{tabular} & \begin{tabular}[c]{@{}c@{}}Dice \\ training set\end{tabular} & \begin{tabular}[c]{@{}c@{}}Dice \\ testing set\end{tabular} & Sensitivity & Specificity & \begin{tabular}[c]{@{}c@{}}Dice \\ test CT scan\end{tabular} \\ \hline
1  & 0.957                                                        & 0.959                                                       & 0.887       & 0.972       & 0.971                                                        \\ 
2  & 0.958                                                        & 0.961                                                       & 0.856       & 0.974       & 0.972                                                        \\ 
3  & 0.953                                                        & 0.951                                                       & 0.915       & 0.967       & 0.966                                                       
\end{tabular}\\

\begin{figure}
\centering
\includegraphics[trim=2.5cm 1.5cm 2cm 1.5cm, clip=true, height=80mm, width=150mm]{Chapter3/mask_results_varying_number_of_connected_layers.png}
\caption{Masks for varying number of connected layers}
\end{figure}

\noindent All three architectures have similar performances on the test CT scan. The architectures with 1 and 2 connected layers in particular that have the highest Dice coefficients of the three have marginally different specificities of 0.972 and 0.974 respectively. The discrepancies between their sensitivities are somewhat more pronounced at 0.887 and 0.856 respectively. Figure whatever similarly shows very little differences in the colour profile of the masks for all three architectures. Thus it seems that adding extra connected layers doesn't make much difference to the performance of the classifier and hence we opt for having 1 fully connected layer in our final architecture.

\subsection{varying number of feature maps}

\noindent We now focus on finding the right number of feature maps for the convolutional layers. In order to keep the computation comparable between the 2 layers, we set the number of feature maps in the second layer to be twice that of the first layer. We tried configurations with the first layer having 16, 32, and 64 feature maps. The summary of the results are shown below.\\

\begin{tabular}{cccccc}
\rowcolor[HTML]{C0C0C0} 
          \begin{tabular}[c]{@{}c@{}}\# Feature \\ Maps \end{tabular}  & \begin{tabular}[c]{@{}c@{}}Dice \\ training set\end{tabular} & \begin{tabular}[c]{@{}c@{}}Dice \\ testing set\end{tabular} & Sensitivity & Specificity & \begin{tabular}[c]{@{}c@{}}Dice \\ test CT scan\end{tabular} \\ \hline
 16  & 0.956                                                        & 0.953                                                       & 0.919       & 0.967       & 0.966                                                        \\ 
 32  & 0.957                                                        & 0.958                                                       & 0.894       & 0.972       & 0.970                                                        \\ 
 64  & 0.959                                                        & 0.958                                                       & 0.904       & 0.971       & 0.970                                                        
\end{tabular}\\

\noindent Yet again there is not much difference between the performance of all three architectures. The two with the highest test Dice coefficients are the ones with the first layer having 32 and 64 feature maps both at 0.970 and very similar sensitivities around 0.9. The one with 16 feature maps yields a slightly lower specificity of 0.967 but a higher sensitivity of 0.919 with an overall lower Dice coefficient of 0.966. The masks show slight differences in the first row of images, with a larger pink region for the 16 feature map architecture. These statistics don't justify having twice as many feature maps, we opt for having 32 feature maps i the first layer of the architecture.

\begin{figure}
\centering
\includegraphics[trim=2.5cm 1.5cm 2cm 1.5cm, clip=true, height=80mm, width=150mm]{Chapter3/mask_results_varying_number_of_feature_maps.png}
\caption{Masks for varying number of feature maps}
\end{figure}

\subsection{varying the number of hidden units}

\noindent We now vary the number of hidden units in the connected layer, trying 100, 200, 500, 1000 hidden units. \\

\begin{tabular}{cccccc}
\rowcolor[HTML]{C0C0C0} 
                 \begin{tabular}[c]{@{}c@{}}\# Hidden \\ Units \end{tabular} & \begin{tabular}[c]{@{}c@{}}Dice \\ training set\end{tabular} & \begin{tabular}[c]{@{}c@{}}Dice \\ testing set\end{tabular} & Sensitivity & Specificity & \begin{tabular}[c]{@{}c@{}}Dice \\ test CT scan\end{tabular} \\ \hline
100   & 0.956                                                        & 0.958                                                       & 0.879       & 0.972       & 0.97                                                         \\ 
200   & 0.958                                                        & 0.957                                                       & 0.878       & 0.971       & 0.97                                                         \\ 
500   & 0.957                                                        & 0.955                                                       & 0.894       & 0.97        & 0.968                                                        \\ 
1000  & 0.957                                                        & 0.958                                                       & 0.892       & 0.972       & 0.97                                                        
\end{tabular}\\

\noindent Here the test statistics across all 4 architectures are very similar with a specificity around 0.97 and a sensitivity around 0.89. Given the tie, we select the simpler model with 100 hidden units.

\begin{figure}
\centering
\includegraphics[trim=2.5cm 1.5cm 2cm 1.5cm, clip=true, height=80mm, width=150mm]{Chapter3/mask_results_varying_number_of_hidden_units.png}
\caption{Masks for varying number of hidden units}
\end{figure}

\subsection{varying the learning rate}

\noindent Tried different learning rates: 0.01, 0.05, 0.1, 0.5. Huge improvement as we increase the learning rate. A learning rate of 1 doesn't converge. We take LR = 0.5.

\begin{tabular}{cccccc}
\rowcolor[HTML]{C0C0C0} 
\begin{tabular}[c]{@{}c@{}}Learning\\ Rate\end{tabular} & \begin{tabular}[c]{@{}c@{}}Dice \\ training set\end{tabular} & \begin{tabular}[c]{@{}c@{}}Dice \\ testing set\end{tabular} & Sensitivity & Specificity & \begin{tabular}[c]{@{}c@{}}Dice \\ test CT scan\end{tabular} \\ \hline
\rowcolor[HTML]{FFFFFF} 
LR 0.01                                                 & 0.956                                                        & 0.957                                                       & 0.885       & 0.971       & 0.97                                                         \\
\rowcolor[HTML]{FFFFFF} 
LR 0.05                                                 & 0.97                                                         & 0.97                                                        & 0.915       & 0.977       & 0.976                                                        \\ 
\rowcolor[HTML]{FFFFFF} 
LR 0.1                                                  & 0.973                                                        & 0.973                                                       & 0.916       & 0.98        & 0.978                                                        \\
\rowcolor[HTML]{FFFFFF} 
LR 0.5                                                  & 0.982                                                        & 0.973                                                       & 0.932       & 0.982       & 0.981                                                       
\end{tabular}

\begin{figure}
\centering
\includegraphics[trim=2.5cm 1.5cm 2cm 1.5cm, clip=true, height=80mm, width=150mm]{Chapter3/mask_results_varying_learning_rate.png}
\caption{Mask for varying learning rate.}
\end{figure}

\subsection{varying the momentum}

\noindent Tried different momentums: 0, 0.01, 0.05, 0.1, 0.5, 1.

\begin{tabular}{cccccc}
\rowcolor[HTML]{C0C0C0} 
Momentum & \begin{tabular}[c]{@{}c@{}}Dice \\ training set\end{tabular} & \begin{tabular}[c]{@{}c@{}}Dice \\ testing set\end{tabular} & Sensitivity & Specificity & \begin{tabular}[c]{@{}c@{}}Dice \\ test CT scan\end{tabular} \\ \hline
\rowcolor[HTML]{FFFFFF} 
0        & 0.982                                                        & 0.973                                                       & 0.906       & 0.982       & 0.981                                                        \\ 
\rowcolor[HTML]{FFFFFF} 
0.05     & 0.979                                                        & 0.973                                                       & 0.916       & 0.98       & 0.979                                                        \\ 
\rowcolor[HTML]{FFFFFF} 
0.1      & 0.983                                                        & 0.976                                                       & 0.9         & 0.984         & 0.982                                                        \\
\rowcolor[HTML]{FFFFFF} 
0.5      & 0.985                                                        & 0.981                                                       & 0.83        & 0.988        & 0.985                                                       
\end{tabular}

\begin{figure}
\centering
\includegraphics[trim=2.5cm 1.5cm 2cm 1.5cm, clip=true, height=80mm, width=150mm]{Chapter3/mask_results_varying_momentum.png}
\caption{Masks for varying momentum}
\end{figure}

\subsection{varying the activation function}

\noindent We experimented with different types of activation function: ReLU, Tanh, Sigmoid. Doing it with ReLU is better...\\

\begin{tabular}{cccccc}
\rowcolor[HTML]{C0C0C0} 
\begin{tabular}[c]{@{}c@{}}Activation\\ Function\end{tabular} & \begin{tabular}[c]{@{}c@{}}Dice \\ training set\end{tabular} & \begin{tabular}[c]{@{}c@{}}Dice \\ testing set\end{tabular} & Sensitivity & Specificity & \begin{tabular}[c]{@{}c@{}}Dice \\ test CT scan\end{tabular} \\ \hline
\rowcolor[HTML]{FFFFFF} 
ReLU                                                          & 0.984                                                        & 0.978                                                       & 0.87        & 0.986       & 0.984                                                        \\
\rowcolor[HTML]{FFFFFF} 
Tanh                                                          & 0.973                                                        & 0.972                                                       & 0.934       & 0.98        & 0.979                                                        \\
\rowcolor[HTML]{FFFFFF} 
Sigmoid                                                       & 0.958                                                        & 0.958                                                       & 0.941       & 0.968       & 0.967                                                       
\end{tabular}

\begin{figure}
\centering
\includegraphics[trim=2.5cm 1.5cm 2cm 1.5cm, clip=true, height=80mm, width=150mm]{Chapter3/mask_results_varying_activation_function.png}
\caption{Masks for varying activation function}
\end{figure}

\subsection{varying datasets type}

\noindent The first thing we investigate is the sampling method for obtaining the training set. In order to increase the number of non-atrium training examples lying near the boundaries where we expect most of the classification errors to lie, we construct a rectangular area which contains the atrium. The atrium box is constructed by going through all the coordinates of the voxels labeled as being in the atrium, picking the minimum and maximum values in each of the coordinate planes, and possibly adding some padding, this procedure gives us a box containing the atrium. \\

\noindent We train our base CNN on 3 sampling procedures with no atrium box, i.e. all the non-atrium training examples are sampled randomly uniformly, a small atrium box constructed by the procedure above with a padding of 5 pixels in the x and y coordinate directions and of 1 pixel in the z coordinate direction, and finally a large atrium box with a padding of 30 pixels in the x and y coordinates and of 5 pixels in the z coordinate.\\

\noindent The results of the three training runs are shown in Figure whatever. From the testing dice coefficient plot, we get a better classification rate with sampling using an atrium box than no atrium box and particularly with the smaller atrium box. In the segmentation mask, sampling with no atrium box clearly yields more errors in the proximity of the atrium but none away from it whereas there are some errors far away from the atrium from the models trained with the atrium box sampling procedure. This is a consequence of the sampling procedures in both cases. Sampling with an atrium box would naturally yield better segmentation results near the atrium as the proportion of training examples is much higher in those regions than without an atrium box around it.\\

\begin{tabular}{cccccc}
\rowcolor[HTML]{C0C0C0} 
\begin{tabular}[c]{@{}c@{}}Training\\ Dataset\end{tabular} & \begin{tabular}[c]{@{}c@{}}Dice \\ training set\end{tabular} & \begin{tabular}[c]{@{}c@{}}Dice \\ testing set\end{tabular} & Sensitivity & Specificity & \begin{tabular}[c]{@{}c@{}}Dice \\ test CT scan\end{tabular} \\ \hline
\rowcolor[HTML]{FFFFFF} 
No Atrium Box                                              & 0.982                                                        & 0.978                                                       & 0.881       & 0.986       & 0.984                                                        \\
\rowcolor[HTML]{FFFFFF} 
Small Atrium Box                                           & 0.96                                                         & 0.987                                                       & 0.838       & 0.994       & 0.991                                                        \\
\rowcolor[HTML]{FFFFFF} 
Large Atrium Box                                           & 0.963                                                        & 0.983                                                       & 0.871       & 0.989       & 0.987                                                       
\end{tabular}

\subsection{varying the data size}

\begin{figure}
\centering
\includegraphics[trim=2.5cm 1.5cm 2cm 1.5cm, clip=true, height=80mm, width=150mm]{Chapter3/mask_results_varying_training_dataset.png}
\caption{Masks for varying training dataset}
\end{figure}

\noindent Tried a number of dataset sizes: 400000, 1000000, 3000000







\chapter{Conclusions and Further Work}

\cleardoublepage
\phantomsection
\addcontentsline{toc}{chapter}{\bibname} % Add an entry for the Bibliography in the Table of Contents

\bibliography{my_bibliography}
\bibliographystyle{plain}

\cleardoublepage 

\include{Appendix}

\end{document}