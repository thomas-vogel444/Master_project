\chapter{Experimental Results}

\section{The Data}

\noindent The aim of this thesis is to implement, and then fine-tune, a CNN to classify voxels of chest computerised tomography (CT) scan as being either in the part of the heart called the atrium or outside it. The atrium is composed of two chambers called the right and left atrium which are the two entry points of the blood into the heart, one returning to the heart to complete the cardiovascular cycle through the body and the other coming back from the lungs after being oxygenated and purified of toxins.\\

\noindent The data from which the training and testing datasets are generated comprise 27 CT scans. CT scans are 3 dimensional grey scale images, generally of size 480*480*50, generated via computers combining many X-ray images taken from different angles to produce cross-sectional images of specific body parts. They are stored as DICOM files, DICOM standing for Digital Imaging and Communications in Medicine which is a standard for handling, storing, printing, and transmitting information in medical imaging. Each has been segmented by trained radiologists at St Thomas's hospital, the results of which are stored in Nearly Raw Raster Data (NRRD) files, a standard format for storing raster data. These are 3D arrays of the same dimension, each entry being either a 1 or 0 indicating whether the corresponding voxel in the CT scan is in the Atrium or not. 

\section{Generating the datasets: The Tri-Planar Method}

\noindent Classifying the voxels require building a set of input vectors each containing enough local and global information to allow the Neural Network to learn effectively. One way of providing 3 dimensional information is to use the tri-planar method. This consists in generating 3 perpendicular square patches of a given dimension in the transversal, saggital and coronal planes centred at the voxel of interest as shown in Figure whatever. This standard technique has been found to give competitive results compared to using 3D patches while being much more computationally and memory efficient [reference the paper that started it on knee cartilage]. In addition, we use a multiscale approach where we add 3 more input channels composed of compressed patches that are originally 5 times larger than the above set of patches, resized to the same size to provide global information about the surroundings of the concerned voxel.\\

\noindent Each patch is fed into a different input channel of the Convolutional Neural Network with one or more convolutional layers. The outputs of those channels are then feed as inputs to a set of connected layers itself connected to a classifying layer as output layer.

\begin{figure}
\centering
\begin{minipage}{0.45\textwidth}
\centering
\includegraphics[trim=2cm 8cm 2cm 8cm, clip=true, height=60mm]{Chapter3/example_slice.pdf}
\end{minipage}\hfill
\begin{minipage}{0.45\textwidth}
\centering
\includegraphics[trim=2cm 8cm 2cm 8cm, clip=true, height=60mm]{Chapter3/triplanar.pdf}
\end{minipage}
\caption{Left: grey scale slices from a CT scan taken in the transversal, saggital and coronal planes. Right: illustration of the triplanar}
\end{figure}

\section{Implementation Details}

\subsection{Libraries}

\noindent To train Neural Networks, we use Torch, an open-source library maintained by Facebook in Lua aiming to provide a Matlab-like environment for scientific computing, along with a number of dependent libraries (nn, cunn, cudnn, fbcunn) to train the convolutional networks on single and multiple GPUs. In addition, we use Python and a number of its libraries to handle all the logistics ranging from generating the datasets to producing plots of segmentation results.

\subsection{Computer facilities}

\noindent The training of the CNNs were conducted alternatively on one of 2 multi-GPU clusters, named montana-nvidia and montana-k80, kindly provided by Prof. Montana's. montana-nvidia consists of 24 cores with 129 Gb of memory connected to two NVIDIA Tesla K40m and two Tesla K20Xm while montana-k80 on the other hand has 56 cores with 258 Gb of memory supported by 8 of NVIDIA's Testla K80. 

\section{Model Selection}

\subsection{General approach for model selection}

\noindent For the purpose of model selection, we allocated 20 of the 27 CT scans to the training set and the remaining 7 to a testing set. The training set is composed of 400000 training examples equally divided among the training CT scans, half being in the atrium and the other half outside it. The validation set is composed of 200000 testing examples equally divided among the testing CT scans. Furthermore at the end of training, each model fully segments a test CT scan from the testing set to provide performance statistics. In particular, from this segmentation image, we evaluate the model's sensitivity, which is the percentage of correctly classified atrium voxels, specificity, which is the percentage of correctly classified non-atrium voxels, and overall classification rate also known as the Dice coefficient, which is calculated by evaluating the proportion of correctly classified voxels in the segmented image. Model selection will be conducted on the basis of these three test statistics.

Additionally, 3 transversal mask images are generated from the segmented CT scan, where the grey CT scan image is overlaid with colours representing the error status of a given voxel. These masks provide a visual element to the performance of the models. The colour codes are:

\begin{itemize}
	\item Green: correctly classified atrium voxel
	\item Blue: correctly classified non-atrium voxel
	\item Red: incorrectly classified atrium voxel
	\item Pink: incorrectly classified non-atrium voxel
\end{itemize}

\noindent We start off our model selection with a CNN comprised of an input layer with 6 channels with patches of size 32*32, 2 convolutional layers with 32 and 64 feature maps respectively, 2 fully connected layers with 1000 and 500 hidden units each respectively, and a logsoftmax output layer giving log probabilities. The training parameters are composed of a learning rate of 0.01, a momentum rate set to 0, and a mini-batch size set to 6000 examples. We will be varying in turn the following hyper-parameters while at each stage keeping the others constant.\\

\begin{itemize}
	\item the number of convolutional layers.
	\item the number of connected layers.
	\item the number of feature maps in the chosen number of convolutional layers.
	\item the number of hidden units in the chosen number of connected layers.
	\item the type of activation function (ReLU, Tanh or Sigmoid).
	\item the type of subsampling function (maximum or average pooling).
	\item the learning rate.
	\item the momentum.
\end{itemize}

\subsection{varying number of convolutional layers}

\noindent To select the number of convolutional layers, we train 4 CNNs with architectures starting with the following convolutional layers:

\begin{itemize}
	\item Input (6*32*32) => Conv layer (32*28*28) => 2*2 MaxPooling filter (32*14*14) 
	\item Input (6*32*32) => Conv layer (32*28*28) => 2*2 MaxPooling filter (32*14*14) => Conv layer (64*10*10) => 2*2 MaxPooling filter (64*5*5)
	\item Input (6*32*32) => Conv layer (32*28*28) => Conv layer (32*24*24) => 2*2 MaxPooling filter (32*12*12) => Conv layer (64*8*8) => 2*2 MaxPooling filter (64*4*4)
	\item Input (6*32*32) => Conv layer (32*28*28) => Conv layer (32*24*24) => 2*2 MaxPooling filter (32*12*12) => Conv layer (64*8*8) => Conv layer (64*4*4) => 2*2 MaxPooling filter (64*2*2)
\end{itemize}

\noindent These convolutional layers are then followed by two fully connected layers with 1000 and 500 hidden layers each. The following table gives the results for each of these architectures trained over 100 epochs.\\

\begin{tabular}{c|c|c|c|c|c}
\rowcolor[HTML]{C0C0C0} 
         & \begin{tabular}[c]{@{}c@{}}Dice \\ training set\end{tabular} & \begin{tabular}[c]{@{}c@{}}Dice \\ testing set\end{tabular} & Sensitivity & Specificity & \begin{tabular}[c]{@{}c@{}}Dice \\ test CT scan\end{tabular} \\ \hline
1 Conv L & 0.959                                                        & 0.956                                                       & 0.913       & 0.970       & 0.969                                                        \\ \hline
2 Conv L & 0.958                                                        & 0.961                                                       & 0.734       & 0.989       & 0.984                                                        \\ \hline
3 Conv L & 0.956                                                        & 0.960                                                       & 0.762       & 0.987       & 0.983                                                        \\ \hline
4 Conv L & 0.959                                                        & 0.956                                                       & 0.881       & 0.973       & 0.972                                                       
\end{tabular}\\

\noindent The three key statistics are the sensitivity, the specificity and the Dice coefficient over the entire test CT scan. As only 1.8\% of its voxels are in the atrium, the specificity has an overbearing effect on the test Dice coefficient and thus takes priority when selecting models. The table above shows a tradeoff between the two across the 4 architectures. While the 1 convolutional layer architecture has the highest sensitivity at 0.913, it also yields the lowest specificity at 0.970, test statistics very close to the 4 convolutional layer architecture. The architectures with 2 and 3 convolutional layers have similar test statistics with relatively low specificities at 0.734 and 0.762 respectively and high specificities at 0.989 and 0.987 respectively.\\

\begin{figure}
\centering
\includegraphics[trim=2.5cm 1.5cm 2cm 1.5cm, clip=true, height=80mm, width=150mm]{Chapter3/mask_results_varying_number_of_convolutional_layers.png}
\caption{Masks for varying number of convolutional layers}
\end{figure}

\noindent Figure whatever brings a visual element to this comparison. Each column shows the segmentation of 3 different transversal slices of the test CT scan image. The first and fourth columns corresponding to segmentations by the architectures with 1 and 4 convolutional layers respectively show much larger pink patches, corresponding to higher rates of false positives than the other two columns but with smaller red regions corresponding to lower rates of false negatives, corroborating the story told by the sensitivity and specificity in the table above. There is not much difference between the masks of the second and third layers.\\

\noindent We choose to go with having a higher specificity at this stage and thus select an architecture with 2 convolutional layers. 

\subsection{varying number of connected layers}

\noindent Having settled on an architecture with 2 convolutional layers, we now train 3 CNNs with 1, 2, and 3 fully connected layers each starting with 1000 hidden units and halving the number of hidden units for each additional layer. Hence the CNN with 3 fully connected layers has 1000, 500, and 250 hidden units in each layer respectively. The results are shown in the following table.\\

\begin{tabular}{c|c|c|c|c|c}
\rowcolor[HTML]{C0C0C0} 
              & \begin{tabular}[c]{@{}c@{}}Dice \\ training set\end{tabular} & \begin{tabular}[c]{@{}c@{}}Dice \\ testing set\end{tabular} & Sensitivity & Specificity & \begin{tabular}[c]{@{}c@{}}Dice \\ test CT scan\end{tabular} \\ \hline
1 Connected L & 0.957                                                        & 0.959                                                       & 0.887       & 0.972       & 0.971                                                        \\ \hline
2 Connected L & 0.958                                                        & 0.961                                                       & 0.856       & 0.974       & 0.972                                                        \\ \hline
3 Connected L & 0.953                                                        & 0.951                                                       & 0.915       & 0.967       & 0.966                                                       
\end{tabular}\\

\begin{figure}
\centering
\includegraphics[trim=2.5cm 1.5cm 2cm 1.5cm, clip=true, height=80mm, width=150mm]{Chapter3/mask_results_varying_number_of_connected_layers.png}
\caption{Masks for varying number of connected layers}
\end{figure}

\noindent All three architectures have similar performances on the test CT scan with specificities around 0.97 and sensitivities around 0.9. Figure whatever shows very little differences in the colour profile of the masks for all three architectures. Thus it seems that adding extra connected layers doesn't make much difference to the performance of the classifier and hence we opt for having 1 fully connected layer in our final architecture.

\subsection{varying number of feature maps}

\noindent We now focus on finding the right number of feature maps for the convolutional layers. In order to keep the computation comparable between the 2 layers, we set the number of feature maps in the second layer to be twice that of the first layer. We tried configurations with the first layer having 16, 32, and 64 feature maps. The summary of the results are shown below.\\

\begin{tabular}{c|c|c|c|c|c}
\rowcolor[HTML]{C0C0C0} 
            & \begin{tabular}[c]{@{}c@{}}Dice \\ training set\end{tabular} & \begin{tabular}[c]{@{}c@{}}Dice \\ testing set\end{tabular} & Sensitivity & Specificity & \begin{tabular}[c]{@{}c@{}}Dice \\ test CT scan\end{tabular} \\ \hline
With 16 FM  & 0.956                                                        & 0.953                                                       & 0.919       & 0.967       & 0.966                                                        \\ \hline
With 32 FM  & 0.957                                                        & 0.958                                                       & 0.894       & 0.972       & 0.970                                                        \\ \hline
With 64 FM  & 0.959                                                        & 0.958                                                       & 0.904       & 0.971       & 0.970                                                        
\end{tabular}\\

\noindent Yet again there is not much difference between the performance of all four architectures. The specificities and sensitivities are around 0.97 and 0.9 respectively. We thus settle with having a first layer with 32 feature maps.

\begin{figure}
\centering
\includegraphics[trim=2.5cm 1.5cm 2cm 1.5cm, clip=true, height=80mm, width=150mm]{Chapter3/mask_results_varying_number_of_feature_maps.png}
\caption{Masks for varying number of feature maps}
\end{figure}

\subsection{varying the number of hidden units}

\noindent We now vary the number of hidden units in the connected layer, trying 100, 200, 500, 1000 hidden units. \\

\begin{tabular}{c|c|c|c|c|c}
\rowcolor[HTML]{C0C0C0} 
                  & \begin{tabular}[c]{@{}c@{}}Dice \\ training set\end{tabular} & \begin{tabular}[c]{@{}c@{}}Dice \\ testing set\end{tabular} & Sensitivity & Specificity & \begin{tabular}[c]{@{}c@{}}Dice \\ test CT scan\end{tabular} \\ \hline
100 Hidden Units  & 0.956                                                        & 0.957                                                       & 0.902       & 0.969       & 0.967                                                        \\ \hline
200 Hidden Units  & 0.955                                                        & 0.957                                                       & 0.913       & 0.968       & 0.967                                                        \\ \hline
500 Hidden Units  & 0.956                                                        & 0.953                                                       & 0.921       & 0.967       & 0.966                                                        \\ \hline
1000 Hidden Units & 0.955                                                        & 0.953                                                       & 0.921       & 0.967       & 0.966                                                       
\end{tabular}\\

\noindent Again the statistics across all 4 cases are very similar except for the increasing sensitivity for the 500 and 1000 hidden unit architectures. We thus select to have 500 hidden units in the fully connected layer.

\begin{figure}
\centering
\includegraphics[trim=2.5cm 1.5cm 2cm 1.5cm, clip=true, height=80mm, width=150mm]{Chapter3/mask_results_varying_number_of_hidden_units.png}
\caption{Masks for varying number of hidden units}
\end{figure}

\subsection{varying the activation function}

\noindent We experimented with different types of activation function: ReLU, Tanh, Sigmoid. Doing it with ReLU is better...\\

\begin{tabular}{c|c|c|c|c|c}
\rowcolor[HTML]{C0C0C0} 
        & \begin{tabular}[c]{@{}c@{}}Dice \\ training set\end{tabular} & \begin{tabular}[c]{@{}c@{}}Dice \\ testing set\end{tabular} & Sensitivity & Specificity & \begin{tabular}[c]{@{}c@{}}Dice \\ test CT scan\end{tabular} \\ \hline
ReLU    & 0.956                                                        & 0.953                                                       & 0.923       & 0.967       & 0.966                                                        \\ \hline
Tanh    & 0.956                                                        & 0.942                                                       & 0.942       & 0.955       & 0.955                                                        \\ \hline
Sigmoid & 0.673                                                        & 0.987                                                       & 0.0         & 1.0         & 0.982                                                       
\end{tabular}

\begin{figure}
\centering
\includegraphics[trim=2.5cm 1.5cm 2cm 1.5cm, clip=true, height=80mm, width=150mm]{Chapter3/mask_results_varying_activation_function.png}
\caption{Masks for varying activation function}
\end{figure}

\subsection{varying the type of pooling}

\noindent We also tried different types of pooling: Max pooling and average pooling. Average Pooling is better.\\

\begin{tabular}{c|c|c|c|c|c}
\rowcolor[HTML]{C0C0C0} 
                & \begin{tabular}[c]{@{}c@{}}Dice \\ training set\end{tabular} & \begin{tabular}[c]{@{}c@{}}Dice \\ testing set\end{tabular} & Sensitivity & Specificity & \begin{tabular}[c]{@{}c@{}}Dice \\ test CT scan\end{tabular} \\ \hline
Max Pooling     & 0.956                                                        & 0.953                                                       & 0.923       & 0.967       & 0.966                                                        \\ \hline
Average Pooling & 0.953                                                        & 0.95                                                        & 0.937       & 0.964       & 0.963                                                       
\end{tabular}

\begin{figure}
\centering
\includegraphics[trim=2.5cm 1.5cm 2cm 1.5cm, clip=true, height=80mm, width=150mm]{Chapter3/mask_results_varying_pooling_function.png}
\caption{Masks for varying pooling function.}
\end{figure}

\subsection{varying the learning rate}

\noindent Tried different learning rates: 0.01, 0.05, 0.1, 0.5. It seems that 0.1 is best. Having a learning rate of 0.5 doesn't train the network.  

\begin{tabular}{c|c|c|c|c|c}
\rowcolor[HTML]{C0C0C0} 
     & \begin{tabular}[c]{@{}c@{}}Dice \\ training set\end{tabular} & \begin{tabular}[c]{@{}c@{}}Dice \\ testing set\end{tabular} & Sensitivity & Specificity & \begin{tabular}[c]{@{}c@{}}Dice \\ test CT scan\end{tabular} \\ \hline
0.01 & 0.953                                                        & 0.95                                                        & 0.936       & 0.964       & 0.963                                                        \\ \hline
0.05 & 0.968                                                        & 0.958                                                       & 0.935       & 0.972       & 0.971                                                        \\ \hline
0.1  & 0.973                                                        & 0.963                                                       & 0.946       & 0.976       & 0.975                                                        \\ \hline
0.5  & 0.667                                                        & 0.987                                                       & 0.0         & 1.0         & 0.982                                                       
\end{tabular}

\begin{figure}
\centering
\includegraphics[trim=2.5cm 1.5cm 2cm 1.5cm, clip=true, height=80mm, width=150mm]{Chapter3/mask_results_varying_learning_rate.png}
\caption{Mask for varying learning rate.}
\end{figure}

\subsection{varying the momentum}

\noindent Tried different momentums: 0, 0.01, 0.05, 0.1, 0.5, 1.

\begin{tabular}{c|c|c|c|c|c}
\rowcolor[HTML]{C0C0C0} 
     & \begin{tabular}[c]{@{}c@{}}Dice \\ training set\end{tabular} & \begin{tabular}[c]{@{}c@{}}Dice \\ testing set\end{tabular} & Sensitivity & Specificity & \begin{tabular}[c]{@{}c@{}}Dice \\ test CT scan\end{tabular} \\ \hline
0    & 0.972                                                        & 0.961                                                       & 0.951       & 0.975       & 0.974                                                        \\ \hline
0.05 & 0.972                                                        & 0.965                                                       & 0.933       & 0.977       & 0.976                                                        \\ \hline
0.1  & 0.972                                                        & 0.962                                                       & 0.945       & 0.975       & 0.975                                                        \\ \hline
0.5  & 0.973                                                        & 0.969                                                       & 0.906       & 0.979       & 0.978                                                       
\end{tabular} 

\begin{figure}
\centering
\includegraphics[trim=2.5cm 1.5cm 2cm 1.5cm, clip=true, height=80mm, width=150mm]{Chapter3/mask_results_varying_momentum.png}
\caption{Masks for varying momentum}
\end{figure}



\subsection{varying datasets type}

\noindent The first thing we investigate is the sampling method for obtaining the training set. In order to increase the number of non-atrium training examples lying near the boundaries where we expect most of the classification errors to lie, we construct a rectangular area which contains the atrium. The atrium box is constructed by going through all the coordinates of the voxels labeled as being in the atrium, picking the minimum and maximum values in each of the coordinate planes, and possibly adding some padding, this procedure gives us a box containing the atrium. \\

\noindent We train our base CNN on 3 sampling procedures with no atrium box, i.e. all the non-atrium training examples are sampled randomly uniformly, a small atrium box constructed by the procedure above with a padding of 5 pixels in the x and y coordinate directions and of 1 pixel in the z coordinate direction, and finally a large atrium box with a padding of 30 pixels in the x and y coordinates and of 5 pixels in the z coordinate.\\

\noindent The results of the three training runs are shown in Figure whatever. From the testing dice coefficient plot, we get a better classification rate with sampling using an atrium box than no atrium box and particularly with the smaller atrium box. In the segmentation mask, sampling with no atrium box clearly yields more errors in the proximity of the atrium but none away from it whereas there are some errors far away from the atrium from the models trained with the atrium box sampling procedure. This is a consequence of the sampling procedures in both cases. Sampling with an atrium box would naturally yield better segmentation results near the atrium as the proportion of training examples is much higher in those regions than without an atrium box around it.\\

\subsection{varying the data size}

\noindent Tried a number of dataset sizes: 400000, 1000000, 3000000






